\section{Project Milestone Plan}

\textit{Technical Writer Note: Standardizing section formats across iterations and adding cross-references to relevant documents.}

\begin{figure}[t!]
\centering
\includegraphics[width=0.9\textwidth]{placeholder}
\caption{Project Timeline Overview}
\label{fig:timeline}
\end{figure}

The project follows a structured approach across four iterations, with a preceding pre-iteration phase. Each phase maintains consistent tracking of goals, deliverables, knowledge development, and organizational structure.

\textit{Fredrik: Adding quantitative data from our tracking systems to support the documented progress.}

\subsection*{Phase Overview}
\addcontentsline{toc}{subsection}{Phase Overview}
\begin{table}[h!]
\begin{tabularx}{\textwidth}{>{\raggedright\arraybackslash}X>{\raggedright\arraybackslash}X}
\toprule
\textbf{Phase} & \textbf{Key Focus} \\
\midrule
Pre-Iteration & Foundation \& Planning \\
Iteration 1 & Core Development Initiation \\
Iteration 2 & System Integration \\
Iteration 3 & Feature Completion \\
Iteration 4 & Quality Assurance \\
\bottomrule
\end{tabularx}
\caption{Project Phase Overview}
\label{tab:phase-overview}
\end{table}

\subsection{Pre-Iteration Phase}
\noindent\rule{\textwidth}{0.4pt}

The pre-iteration phase established our project foundations through systematic role assignments and initial team organization. Team members participated in the Axis crash course, gaining familiarity with ACAP development, hardware connections, and documentation systems. 

Customer needs were gathered through direct interviews with Axis representatives. Initial team organization implemented Gap Analysis principles for cross-functional team formation. The tollgate meeting presented our high-level timeline alongside Chang's architecture overview. User interface concepts were visualized through our PanoraGuard Figma prototype.

\subsection{Iteration 1}
\noindent\rule{\textwidth}{0.4pt}

\subsubsection*{Goals and Vision}
\addcontentsline{toc}{subsubsection}{Goals and Vision}
The first iteration centered on initiating core development activities. Development teams began implementing initial functions while analysts worked on structuring requirements and connecting them to user stories. The testing team started formulating the Quality Assurance plan, while parallel work began on identifying and documenting project risks.

\subsubsection*{Deliverables}
\addcontentsline{toc}{subsubsection}{Deliverables}
This iteration produced three foundational documents: the Requirements Specification, Quality Assurance plan, and Risk Management plan. Each document underwent initial review cycles and established baseline versions for future refinement.

\begin{center}
\noindent\fbox{\begin{minipage}{0.97\textwidth}
\textbf{Knowledge Development:} The front-end development team engaged in React training, establishing the technical foundation for our user interface implementation.
\end{minipage}}
\end{center}

\subsubsection*{Organizational Structure}
\addcontentsline{toc}{subsubsection}{Organizational Structure}
Our organizational approach evolved from the initial Gap Analysis structure to a developer division-based arrangement of cross-functional teams. This adjustment aimed to optimize development workflows and team communication.

\subsection{Iteration 2}
\noindent\rule{\textwidth}{0.4pt}

\subsubsection*{Goals and Vision}
\addcontentsline{toc}{subsubsection}{Goals and Vision}
The second iteration prioritized system integration. Work focused on connecting requirements to development tasks and establishing links between frontend and backend components. Key objectives included implementing user authentication, displaying alarms with snapshots, and deploying the system to Azure. Pipeline implementation became a central focus for ensuring automated testing and build processes.

\begin{center}
\noindent\fbox{\begin{minipage}{0.97\textwidth}
\textbf{Integration Milestones:}
\vspace{1mm}
\begin{description}
\item[Frontend-Backend Connection] User authentication and alarm display
\item[Deployment] Azure infrastructure setup
\item[Pipeline] Automated testing and build system
\end{description}
\end{minipage}}
\end{center}

\subsubsection*{Knowledge Development}
\addcontentsline{toc}{subsubsection}{Knowledge Development}
Two analysts began their transition to development roles through a structured developer training program.

\textit{Fredrik: Adding concrete outcomes from iteration 2 as they are finalized.}

\subsection{Iteration 3 - Current Phase}
\noindent\rule{\textwidth}{0.4pt}

Our current iteration focuses on completing remaining back-end features and their integration with the front-end components. System testing has moved to the forefront of activities, with pipeline refinement continuing as needed based on development requirements.

\vspace{0.5cm}
\begin{center}
\begin{tabular}{|p{0.45\textwidth}|p{0.45\textwidth}|}
\hline
\multicolumn{2}{|c|}{\textbf{Current Focus Areas}} \\
\hline
Back-end Feature Completion & Front-end Integration \\
System Testing Implementation & Pipeline Refinement \\
\hline
\end{tabular}
\end{center}
\vspace{0.5cm}

\subsection{Iteration 4 - Planned}
\noindent\rule{\textwidth}{0.4pt}

The final iteration will serve as a buffer period for completing remaining tasks and ensuring alignment with customer expectations. Testing activities will intensify to ensure product quality meets release standards.

\begin{center}
\noindent\fbox{\begin{minipage}{0.97\textwidth}
\textbf{Final Phase Objectives}
\vspace{1mm}

Quality assurance and testing completion, with focus on customer acceptance criteria. Documentation finalization and product release preparation.
\end{minipage}}
\end{center}

